\section{Approach}
\subsection{Version 1}
\begin{itemize}
    \item[] Implementation 
        \begin{itemize}
            \item[] \textbf{Ist Phase} : creating signature using linked list and cross comparison
            \item[] \textbf{IInd Phase} : comparisons between signature and inputs sequence
        \end{itemize}
    \item[] {Advantage}
        \begin{itemize}
            \item[] Signature was always linear
            \item[] Signature required less space
            \item[] Easier computation and used link list
        \end{itemize}
    \item[] {Disadvantage}
        \begin{itemize}
            \item[] Signature followed the smallest input
            \item[] Signature soon became inadequate as it was too small
            \item[] Did not consider the system calls that are rejected if it didn't exist in both. Hence had to modify unique
        \end{itemize}
    \item[] {Points to be noted}
        \begin{itemize}
            \item[] Signature needs to include branches not done during learning phase
            \item[] System calls like mutex, poll and futex repeated many times hence were not considered during signature creation
        \end{itemize}
\end{itemize}
\subsection{Version 2}
\begin{itemize}
    \item[] {Signature now includes}
        \begin{itemize}
            \item[] A backbone sequence is created by finding the common sequence of system calls in the input dataset
            \item[] Rest of the elements are added to the signature in the forms of pairs as per their occurrences in the input sequence after processing
            \item[] Unique list of system calls to identify those system calls that are never used
        \end{itemize}
\end{itemize}
