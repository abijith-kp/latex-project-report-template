\section{Approach}
\subsection{Version 1}
    \begin{itemize}
        \item[] \textbf{Ist Phase} Preprocessing \\ Creating signature using linked list and cross comparison over set of sequence of system calls.
        \item[] \textbf{IInd Phase} Processing \\ Detecting malware by comparisons of set of signature and input sequence of system calls.
    \end{itemize}
\subsection{Version 2}
\begin{itemize}
    \item[] {New signature design:}
        \begin{itemize}
            \item[*] A backbone sequence is created by finding the common sequence of system calls in the input dataset.
            \item[*] Rest of the elements are added to the signature in the form of pairs as per their occurrences in the input sequence after processing.
            \item[*] Unique list of system calls to identify those system calls that are never used.
        \end{itemize}
\end{itemize}
\begin{itemize}
    \item[] {Advantage}
        \begin{itemize}
            \item[*] Showed that an application does have a finitely large backbone which would be present in any run of the application.
            \item[*] Signature database required less space.
            \item[*] Easier computation and used link list for implementation.
            \item[*] Signature was always linear.
        \end{itemize}
    \item[] {Disadvantage}
        \begin{itemize}
            \item[*] Signature followed the smallest input sequence of system calls.
            \item[*] Signature converged and became inadequate as it was too small.
            \item[*] The signature did not consider the system calls that are rejected while cross comparing if it didn't exist in either of them.
            \item[*] Huge chances of false positive and hence the need to modify the preliminary checks. \\
        \end{itemize}
    \item[] {TO-DO:}
        \begin{itemize}                                        
            \item[] Signature needs to include branches which was not done during learning phase. Signature needs to incorporate a feedback system so as to develop effective system. Information loss over lesser space complexity will lead to a poor system. All information should be captured during the learning phase. System calls like mutex, poll and futex are repeated many times and hence were not considered during signature creation.
        \end{itemize}
\end{itemize}
Hence an improvised version of the IDS has been designed inculcating the advantages and eliminating the disadvantages from previous versions to the maximum extend possible.
